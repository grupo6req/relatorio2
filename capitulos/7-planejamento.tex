\chapter{Planejamento}
\label{planning}

Visando uma jornada livre de imprevistos e dentro do custo e prazo esperados, foi necessário a utilização de um cronograma refinado para auxiliar a equipe. O cronograma citado, foi divido em três macro atividades, organizando o desenvolvimento do projeto desde o seu início até a sua entrega. Como podemos ver na figura X, as três macro atividades são Iniciação, Elaboração e Construção.

\section{Iniciação}

Na fase de iniciação, a equipe procurou situar-se em relação ao problema e organizar as atividades, definidas pelo cronograma, para obter uma base para o projeto. Como exemplo dessas atividades, estão escolher abordagem e definir o processo.

\section{Elaboração}

Na fase de elaboração, estão localizadas as atividades que são de suma importância para o desenvolvimento da solução proposta, algumas são a elaboração do documento de visão e a priorização de casos de uso.

\section{Construção}

Nessa fase, as atividades estão relacionadas ao desenvolvimento do produto. São utilizadas, portanto, as atividades de engenharia de requisitos também usadas na fase anterior. O resultado da construção é a entrega parcial do produto através da implementação dos casos de uso priorizados pelo projeto.

