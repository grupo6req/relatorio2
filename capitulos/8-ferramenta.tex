\chapter{Ferramenta de Gestão de Requisitos}

“Para ser efetiva, a melhoria de processo de \textit{software} precisa contar com ferramentas que apoiem as atividades previstas nos processos. Sem a agilidade na execução, obtida por meio da adoção de ferramentas, as atividades que permitem visibilidade e controle dos processos podem se tornar um gargalo e, consequentemente, dificultar a institucionalização dos processos.” (MENDES, 2010)

\section{Análise de ferramentas}

\subsection{IBM® Rational® RequisitePro}

O IBM Rational RequisitePro é uma ferramenta de gerenciamento de requisitos desenvolvida pela Rational Software, uma divisão dentro da IBM responsável pela elaboração de \textit{softwares} para auxiliar no desenvolvimento de um projeto, o RequisitePro é uma ferramenta que permite gerenciar e rastrear os requisitos de negócio. 

Esta ferramenta é baseada no modelo iterativo incremental de desenvolvimento de \textit{software}, o RUP, que por sua vez é um modelo criado pela Rational Software. Ela unifica as duas abordagens da gerência de requisitos, a abordagem centrada a documentos e a centrada a banco de dados. Como o RequesitePro tem suporte nativo ao Microsoft Word®, ele consegue sincronizar um banco de dados comercial com o Word, resultando na organização dos requisitos através desta sincronia, evitando um repositório com documentos discordantes. Os principais pontos fortes encontrados na ferramenta, são a percepção automática da quebra de relacionamentos e a criação de visões da matriz de rastreabilidade. 

Alguns empecilhos para a utilização desta ferramenta é a sua compatibilidade apenas com a plataforma Windows, e pelo fato de ser uma ferramenta paga, mesmo tendo um período de testes grátis, o \textit{trial}.

\subsection{Innoslate}

O Innoslate é uma ferramenta voltada ao ciclo de vida completo de um sistema, e não apenas de gerenciamento de requisitos. Em relação ao gerenciamento dos requisitos, esta ferramenta possui várias funções para auxiliá-lo.

Primeiramente ela nos possibilita facilmente editar e revisar os requisitos como uma janela de visualização de requisitos muito intuitiva. Assim como o Rational RequisitePro, o Innoslate nos permite importar os requisitos de outras plataformas como o Microsoft Word®, Microsoft Excel®, IBM® Rational® DOORS®, identificando e criando as relações pai/filho automaticamente. Uma outra \textit{feature} bastante útil é a sua análise de requisitos, que exibe uma barra de qualidade para cada requisito, em porcentagem, além de dar sugestões para o seu aprimoramento. Estas são as principais funções desta ferramenta, mas ela ainda conta com gráficos hierárquicos, controle de versão e um ambiente customizável.

Mesmo o Innoslate sendo uma ferramente paga, ela possui uma função que permite a importação de projetos da mesma. Como ela dispõe de um período de testes (30 dias), o que não acarretaria a grandes problemas ao projeto, pelo fato do time ser composto por 4 integrantes, totalizando 120 dias de utilização. Além de ser uma ferramenta \textit{online}, impedindo assim uma incompatibilidade com sistemas operacionais.

\subsection{Requirements One}

O Requirements One é uma ferramenta de gerência de requisitos voltada para a integração entre o time de desenvolvimento e os outros \textit{stakeholders}. Possui uma ampla gama de funcionalidades, as quais permitem um bom gerenciamento de requisitos, além de possuir diversas integrações possíveis. As funcionalidades que se destacam é o seu modo de planejamento, permitindo o gerente do projeto estar a par de tudo o que está acontecendo, e seu sistema de \textit{issues}, o qual facilita o entendimento da fonte da \textit{issue}, que é de fundamental importância para a sua resolução com o menor tempo possível.

Esta ferramenta, como o Innoslate, é uma plataforma \textit{web} paga, também possuindo um período de testes. Possui uma interface bem agradável e alguns diferenciais bem interessantes, como \textit{wiki} e validação dos requisitos via \textit{feedback}. 

\section{Critérios de avaliação}

Segundo (BEATTY, 2013), há uma série de atributos que devem ser analisados durante a escolha de uma ferramenta de gerenciamento de requisitos, entre eles estão armazenamento de requisitos e seus atributos, rastreabilidade, usabilidade e gestão de mudanças. Com isso em mente, analisou-se alguns desses critérios para escolha da ferramenta.

\begin{itemize}
\item Armazenamento de requisitos e seus atributos: Esta característica visa comparar a manutenção, análise e cadastro de requisitos.
\item Rastreabilidade: A partir dessa característica, a comparação pode ser realizadas a partir da maneira que os requisitos são organizados e se a ferramenta possui funcionalidades para obter uma árvore de rastreabilidade.
\item Usabilidade: Com esta característica, as ferramentas foram analisadas se forneciam uma boa experiência de uso entre o usuário e o serviço, utilizando as boas práticas de iteração humano-computador.
\item Gestão de mudanças: Esta é uma das características que mais se destaca, pois ela busca verificar como os requisitos são modificados e fornecer uma visão da qualidade da elicitação.
\end{itemize}

A análise das ferramentas foi baseada nos critérios supracitados, em que cada ferramenta recebeu uma avaliação entre 0 (péssimo) e 10 (ótimo) para cada item.

\section{Escolha da ferramenta}

A equipe determinou que além das características de avaliação, a ferramenta deveria ser compatível com todos os sistemas operacionais utilizados pela mesma, além de ser uma ferramenta viável, do ponto de vista financeiro. A partir destes itens, a ferramenta escolhida foi a Innoslate por ser a que mais se destacou em relação aos critérios de avaliação e aos ressalvas da equipe.


