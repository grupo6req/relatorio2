\chapter{Elicitação de Requisitos}
\label{elicitation}

\section{Elicitação}

A elicitação é a atividade em que são capturadas as características do produto com base em interação com o cliente. Pode ser considerada a atividade mais importante do processo de Engenharia de Requisitos ~\cite{goguen}, visto que seu objetivo é evitar que se tenha um \textit{software} que atenda aos requisitos capturados mas que não resolve o problema do cliente. ~\cite{pressman}

Não há uma "fórmula mágica" para realizar essa atividade. Existe, no entanto, um conjunto de técnicas que tem como propósito facilitar esse processo. Esse capítulo tem como objetivo explicar como foram usados as técnicas que foram especificadas no primeiro relatório. A interação do grupo com a empresa se deu através do aluno Romenigue Igor Melo Araujo Fernandes, que ocupa a função de diretor de qualidade.

\section{Técnicas}

As técnicas foram escolhidas tendo em vista suas descrições, adequamento das mesmas com o comportamento observado no cliente e a metodologia escolhida para nortear o processo de desenvolvimento.

\subsection{Entrevista}

Entrevistas são encontros formais e informais com os \textit{stakeholders} que são caracterizadas por questionamentos, que podem ser fechadas, onde há um conjunto pré-definido de perguntas, ou abertas, em que não há agenda. Essa técnica é encontrada na maioria dos processos de ER ~\cite{sommerville}. É fundamental, nesse contexto, que o entrevistado tenha total liberdade de expressão nesses encontros.

As entrevistas foram realizadas de forma informal, para acompanhar o estilo das reuniões que foram ocorrendo durante o semestre. Isso é, o processo de elicitação foi mais informal para acompanhar um cliente informal. As perguntas eram realizadas de forma a obter o máximo de informação possível acerca do contexto da empresa e dos problemas enfrentados pela mesma. A primeira entrevista, no entanto, foi realizada formalmente, tendo sido produzido um documento com as informações adquiridas da mesma.


\subsection{Observação}

A observação é uma ferramenta poderosa, no sentido de que permite uma análise do processo por parte da equipe de desenvolvimento, sendo visualizada a realidade de uma situação, em vez de apenas ouvir sobre a mesma por terceiros~\cite{dennis}. Outro aspecto que torna essa forma de levantamento de requisitos muito importante é que, em uma conversa, o interlocutor pode esquecer de informações importantes sobre o processo ao descrevê-lo para a equipe, o que não ocorre no ato de observar enquanto o processo é executado.

Essa técnica entra para complementar as entrevistas. Se na técnica supracitada o objetivo era obter de forma abstrata as soluções do \textit{software}, na observação o objetivo é tornar mais claro esses serviços e analisar de forma crítica as situações de problema vividas pelo cliente para que nenhum detalhe passe despercebido.

Essa técnica não foi utilizada na prática. Isso ocorreu devido à falta de disponibilidade entre os integrantes do grupo no horário em que as reuniões da empresa ocorriam. Segundo o cliente, os encontros da Engrena ocorriam no sábado à tarde, tornando difícil, portanto, a aplicação dessa técnica. Essa técnica foi substituída pela próxima a ser descrita.

\subsection{Quadro Branco}

Essa técnica foi introduzida de forma rápida pelo doutorando Fernando Wanderley durante as aulas de requisitos. Consiste em utilizar um quadro branco com vários pincéis coloridos para se comunicar com o cliente acerca do que o sistema deve fazer. Foi uma grande surpresa para a equipe de desenvolvimento e até para o cliente a eficácia dessa técnica.

A possibilidade do cliente interagir de forma ativa, ao pegar o pincel e desenhar para a equipe o que era esperado do sistema tornou a interação muito mais leve, como um todo. Isso porque, de acordo com o Fernando, é uma forma de comunicação que o cliente consegue entender e concretizar.

\subsection{Cenários}

A técnica de cenário é uma forma de tornar a relação entre as pessoas e as funcionalidades do sistema mais fáceis. É uma descrição de como seria uma interação entre as mesmas e o software, de forma que possam compreender e criticar essa operação~\cite{sommerville}. São úteis na obtenção de detalhes mais aprofundados à requisitos que ainda estão gerais.

Os cenários serão utilizados para reduzir o nível de abstração das técnicas utilizadas anteriormente. Através dos mesmos, é esperado que os requisitos alcancem uma forma mais técnica e tangível, de forma a estar preparada para se tornar, enfim, um caso de uso.

Essa técnica foi utilizada exatamente de acordo com a descrição. Os cenários foram sendo elaborados com o cliente e cada vez mais os requisitos foram ficando mais concretos e claros. Essa técnica teve uma alta complementação em relação a técnica do quadro branco. Isso porque, enquanto os cenários eram descritos, o cliente e a equipe escreviam no quadro os resultados dos cenários. 
