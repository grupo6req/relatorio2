\chapter{Introdução}
\label{introduction}

Segundo Pressman, um processo é um conjunto de atividades, ações e tarefas que são realizadas na elaboração de um produto. Para desenvolver um \textit{Software}, deve-se seguir passos definidos pela equipe, passos esses que compõe o processo de desenvolvimento.~\cite{pressman}

O primeiro relatório tinha como objetivo conter informações acerca do processo escolhido pela equipe de desenvolvimento para elaboração do produto. Nesse relatório, objetiva-se descever como foi o prosseguimento do planejamento através do processo definido. Questões como técnicas de elicitações, refatorações do processo e artefatos produzidos serão todos abordados durante o desenvolvimento deste documento.

\section{Organização do Documento}

Este documento possui a seguinte estrutura:
\begin{itemize}
\item Capítulo 2 - Refatoração: Erros que foram encontrados e corrigidos durante o trabalho;
\item Capítulo 3 - Processo: Breve descrição das atividades do processo escolhido;
\item Capítulo 4 - Técnicas de Elicitação: Comentários sobre o uso das técnicas especificadas no primeiro relatório;
\item Capítulo 5 - Gerência de Requisitos: Explicação sobre as técnicas e atividades relacionadas à gerência de requisitos no trabalho;
\item Capítulo 6 - Conclusão: Considerações finais acerca do trabalho feito em conjunto com a Engrena;
\end{itemize}
