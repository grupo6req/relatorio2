\begin{apendicesenv}
\chapter{Especificação de Regras da Empresa}

\begin{description}
\item [Rotatividade de funcionários] Dentro do contexto da empresa júnior Engrena, existe a política de rotatividade adotado, se espelhando em várias outras empresas que obtiveram sucesso não apenas na satisfação do empregado, retirando o sentimento de monotonia e identificando quais características emergem do funcionário, podendo assim encaminhar o mesmo para uma área mais adequada.

\item [Desempenho dos funcionários] A Engrena possui uma grande dificuldade em identificar e exibir para os diretores como um funcionário está desempenhando o seu papel dentro da empresa, atualmente processando-se através de planilhas no Excel®, desta forma exibindo apenas resultados binários da execução de tarefas.

\item [Identificação de prováveis diretores] Dentro da política da Engrena, existe uma filosofia de promoção de funcionários através de suas habilidades e desempenho. Com o atual cenário de muitos diretores terem se formado, surgiu uma grande demanda para os cargos de diretores, porém com a vigente estrutura de reconhecimento do desempenho, esta filosofia está quase que impossibilitada.

\item [Entrada de pessoas sem experiência técnica] Por se tratar de uma empresa júnior, o ingresso de pessoas na Engrena é feito sem a necessidade obrigatória de muita experiência técnica. Isso é, o único requisito para se tornar um funcionário da empresa é estar disposto a aprender novas coisas frequentemente.

\end{description}




























\chapter{Documento de Visão}

\section{Introdução}

Este artefato tem por objetivo, descrever, para um melhor entendimento de ambas as partes
sobre as necessidades e recursos de nível superior do Engerir . Ele concentra-se nos recursos necessários aos envolvidos e ao público-alvo, além das razões que levam a essas necessidades. Os detalhes de como o EnGerir satisfaz essas necessidades são descritos no caso de uso e nas especificações suplementares.

\section{Posicionamento}
\subsection{Oportunidades de Negócios}

Dentro da disciplina de Engenharia de Requisitos, tivemos a oportunidade de estudar as necessidades da empresa júnior Engrena, a qual era controlar os seus funcionários de uma maneira mais eficiente.
Observou-se também, com base em todos os estudos prévios dentro da disciplina, a necessidade de utilizar das várias técnicas e métodos previstos na metodologia tradicional, para a resolução do problema da Engrena.


\subsection{Descrição do Problema}


\begin{table}[h!]
\centering
\caption{Problemas}
\label{problem}
\begin{tabular}{|p{7cm}|p{7cm}|}
\hline
ID                    & PB 1                                                                                                                  \\ \hline
O problema de         & gerenciar os membros e as atividades dos mesmos                                                                       \\ \hline
afeta                 & os gestores da Engrena                                                                                                \\ \hline
cujo impacto é        & diminuição na transparência das atividades dos membros, além de diminuição da rastreabilidade das entregas acadêmicas \\ \hline
uma boa solução seria & uma ferramenta de gerência completa que permita gerenciar membros e entregas acadêmicas                               \\ \hline
\end{tabular}
\end{table}

\subsection{Sentença de Posição do Produto}

\begin{table}[h!]
\centering
\caption{Sentença do Produto}
\label{product}
\begin{tabular}{|p{7cm}|p{7cm}|}
\hline
Para            & gestores das Engrena                                                                                                         \\ \hline
Que             & necessitam de um melhor gerenciamento na distribuição das atividades de seus membros bem como na obtenção de seus resultados \\ \hline
O               & EnGerir                                                                                                                      \\ \hline
Que             & auxilia no gerenciamento das atividades dos membros da empresa júnior                                                        \\ \hline
Ao contrário de & um software de planilha(como o MS Excel)                                                               \\ \hline
Nosso produto   & irá facilitar o gerenciamento de atividades entre os membros, bem como analisar seus desempenhos                             \\ \hline
\end{tabular}
\end{table}

\section{Descrição dos Envolvidos e dos Usuários}

Nesta seção do documento, pretende-se apresentar os envolvidos e os usuários que farão a interação com a solução aqui descrita. Além disso, haverá uma apresentação geral sobre os papéis de cada envolvido, bem como uma descrição do mesmo.
Dessa forma, pretende-se estabelecer de forma clara quem é o beneficiado da aplicação, além de todos os envolvidos que não serão usuários diretos do sistema, mas que serão afetados pelo mesmo.

\subsection{Resumo dos Envolvidos}

\begin{table}[h!]
\centering
\caption{Envolvidos do projeto}
\label{stake}
\begin{tabular}{{|p{4cm}|p{4cm}|p{4cm}|}}
\hline
Nome                                 & Descrição                                                                              & Responsabilidades                                                                               \\ \hline
Equipe de Desenvolvimento            & Estudantes da Universidade de Brasília, da disciplina de Requisitos de Software        & Desenvolver o processo e aplicá-lo com o cliente estabelecido pela disciplina                   \\ \hline
Gustavo Sabino                       & Estudante da Universidade de Brasília, monitor da disciplina de Requisitos de Software & Auxiliar a equipe de desenvolvimento na tarefa de desenvolver e aplicar o processo              \\ \hline
Professora Elaine Venson             & Professora que ministra a disciplina de Requisitos de Software                         & Auxiliar a equipe de desenvolvimento e orientar sobre a implementação                           \\ \hline
Romenigue Igor Melo Araujo Fernandes & Representante da Engrena responsável por interagir com a equipe de desenvolvimento     & Descrever a empresa e o problema, sendo fonte de requisitos para o desenvolvimento da aplicação \\ \hline
\end{tabular}
\end{table}

\subsection{Resumo dos Usuários}

\begin{table}[h!]
\centering
\caption{Usuários do projeto}
\label{users}
\begin{tabular}{|p{4cm}|p{4cm}|p{4cm}|}
\hline
Nome                            & Descrição                                                                                          & Responsabilidades                                                                                                                                             \\ \hline
Diretores e Gestores da Engrena & Líderes de direções da empresa, responsáveis por supervisionar o andamento da empresa como um todo & No sistema, os diretores serão capazes de acompanhar o desempenho de áreas e usuários individuais. Além disso, poderão criar atividades para outros usuários. \\ \hline
Funcionários da Engrena         & Funcionários gerais da empresa, responsáveis por desempenhar funções estabelecidas pelos diretores & Verificar a existência de atividades atribuídas a si para ser capaz de fazê-las.                                                                              \\ \hline
\end{tabular}
\end{table}

\subsection{Ambiente do Usuário}

O \textit{software} será implementado via \textit{web}, sendo portanto possível a utilização pelos navegadores mais difundidos, entre eles:

\begin{itemize}
\item Google Chrome
\item Internet Explorer 9 ou superior
\item Microsoft Edge
\item Mozilla Firefox
\item Safari
\item Outros
\end{itemize}

\subsection{Principais Necessidades dos Usuários ou dos Envolvidos}

\begin{table}[h!]
\centering
\caption{Necessidades}
\label{needs}
\begin{tabular}{|p{1cm}|p{3cm}|p{2cm}|p{3cm}|p{2cm}|p{3cm}|}
\hline
ID    & Necessidade                                                                                                                & Prioridade & Preocupações                                                                & Solução Atual   & Soluções Propostas                                                                                                           \\ \hline
NE1.1 & Verificar desempenho de áreas e usuários individuais dentro da empresa. Também consultar tal informação relativo ao tempo. & Alta       & Perda de produtividade nas atividades executadas pela empresa               & Planilhas Excel & Agregação das informações relativas à entregas de atividades em forma de gráficos                                            \\ \hline
NE1.2 & Criar tarefas em um ambiente virtual compartilhado e assinalar funcionários para sua execução, com atributos               & Alta       & Falta de uma maneira de criar tarefas e rastreá-las de forma efetiva depois & Planilhas Excel & Um mecanismo capaz de criar tarefas e atribuir para determinados usuários, com os devidos atributos relativos à tais tarefas \\ \hline
\end{tabular}
\end{table}

\subsection{Características}

\begin{table}[h!]
\centering
\caption{Características encontradas}
\label{looks}
\begin{tabular}{|p{4cm}|p{4cm}|}
\hline
ID       & Características                            \\ \hline
CA 1.1.1 & Agregar informações                        \\ \hline
CA 1.1.2 & Mostrar desempenho de membros graficamente \\ \hline
CA 1.2.1 & Criar tarefas em um ambiente virtual       \\ \hline
CA 1.2.2 & Verificar tarefas completadas              \\ \hline
CA 1.2.3 & Criar áreas para alocamento de membros     \\ \hline
\end{tabular}
\end{table}

\subsection{Alternativas e Concorrência}

\begin{description}

\item [Moodle] Ferramenta amplamente utilizada na Universidade de Brasília. Permite o envio de atividades e atribuição de nota às mesmas, que é um dos desejos do cliente. Porém, não há a função de visualização de desempenho por gráfico.

\item [Microsoft Excel] Forma atual de gerenciamento de desempenho e de atividades da empresa. Não é muito eficiente, porém é uma alternativa barata e que os diretores já tem certa experiência.

\item [Trello] Ferramenta para gerenciamento de projetos. Forte concorrente por ser bem discriminado entre os estudantes e no mercado, o Trello é uma excelente ferramenta para gerenciamento, todavia ela não possui uma feature para gerar gráficos, o que é uma necessidade fundamental para o nosso cliente.

\end{description}

\section{Visão Geral do Produto}

\subsection{Perspectiva do Produto}

O sistema deverá auxiliar no aumento da produtividade da empresa júnior por meio de um sistema de gerenciamento de pessoas eficiente e descomplicado. O desempenho das direções e dos funcionários da empresa poderão ser vistos em gráficos que representarão a quantidade de tarefas que tal direção ou funcionário fez de todas que tinha para fazer.
Além disso, outras possibilidades estão inclusas na solução proposta, como a criação e atribuição de tarefas por parte dos diretores aos funcionários da empresa, podendo inclusive atribuir notas à essas tarefas. Por parte dos funcionários, os mesmos poderão visualizar quais tarefas foram designadas à eles, permitindo um maior planejamento e evitando situações de esquecimento de atividades.

\subsection{Suposições e Dependências}

Por se tratar de um sistema web, algumas dependências devem ser atingidas para que o mesmo funcione corretamente:

\begin{itemize}
\item Dispositivo eletrônico com \textit{software} de \textit{browser};
\item Disponibilidade de servidores;
\item Acesso à \textit{internet}.
\end{itemize}

\section{Recursos do Produto}

\begin{itemize}
\item Capacidade de gerenciar tarefas
\item Capacidade de alocar tarefas para usuários
\item Capacidade de fornecer relatórios e estatísticas sobre o desempenho de um membro
\item Capacidade de assinalar uma tarefa como terminada
\item Capacidade de gerenciar as pessoas que entram e saem da empresa
\item Capacidade de armazenar os dados dos membros
\item Capacidade de gerenciar as contas de acesso ao sistema, de forma que elas tenham diferentes níveis de acesso a este
\end{itemize}

\section{Outros Requisitos do Produto}

Será uma aplicação web, que deverá estar disponível 24 horas por dia, durante todos os dias da semana.
Ela não precisará ter um alto desempenho, como por exemplo, abrir a página em menos de 2 segundos.

\end{apendicesenv}
