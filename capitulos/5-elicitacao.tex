\chapter{Elicitação de Requisitos}
\label{elicitation}

\section{Elicitação}

É a atividade em que a equipe de desenvolvimento interage com clientes e usuários finais com o fim de obter mais informações sobre o que o sistema deve fazer, restrições de hardware, domínio da aplicação e outras informações~\cite{sommerville}. A elicitação pode ser considerada a atividade mais importante do processo de ER ~\cite{goguen}, visto que sua função é evitar a situação de ter um \textit{software} funcional e elegante que não resolva os problemas errados, não atendendo as necessidades do cliente.~\cite{pressman}

Não há uma forma definida e fechada para completar essa atividade, havendo na verdade um conjunto de técnicas que permitem facilitar a atividade de explicitar as necessidades do cliente.  Cada técnica apresenta um caso de uso em que é mais vantajosa e tem suas limitações, de forma que variáveis como comportamento do cliente, metodologia e habilidade de expressão do cliente devem ser levadas em conta no processo de escolha das técnicas.

\section{Técnicas}

As técnicas foram escolhidas tendo em vista suas descrições, adequamento das mesmas com o comportamento observado no cliente e a metodologia escolhida para nortear o processo de desenvolvimento.

\subsection{Entrevista}

Entrevistas são encontros formais e informais com os \textit{stakeholders} que são caracterizadas por questionamentos, que podem ser fechadas, onde há um conjunto pré-definido de perguntas, ou abertas, em que não há agenda. Essa técnica é encontrada na maioria dos processos de ER ~\cite{sommerville}. É fundamental, nesse contexto, que o entrevistado tenha total liberdade de expressão nesses encontros.

As entrevistas que serão realizadas consistirão em perguntas simples sobre o contexto e necessidades da empresa. É pretendido, portanto, traçar de forma abstrata os serviços desejados para a solução buscada pelo cliente.

\subsection{Observação}

A observação é uma ferramenta poderosa, no sentido de que permite uma análise do processo por parte da equipe de desenvolvimento, sendo visualizada a realidade de uma situação, em vez de apenas ouvir sobre a mesma por terceiros~\cite{dennis}. Outro aspecto que torna essa forma de levantamento de requisitos muito importante é que, em uma conversa, o interlocutor pode esquecer de informações importantes sobre o processo ao descrevê-lo para a equipe, o que não ocorre no ato de observar enquanto o processo é executado.

Essa técnica entra para complementar as entrevistas. Se na técnica supracitada o objetivo era obter de forma abstrata as soluções do \textit{software}, na observação o objetivo é tornar mais claro esses serviços e analisar de forma crítica as situações de problema vividas pelo cliente para que nenhum detalhe passe despercebido.

\subsection{Cenários}

A técnica de cenário é uma forma de tornar a relação entre as pessoas e as funcionalidades do sistema mais fáceis. É uma descrição de como seria uma interação entre as mesmas e o software, de forma que possam compreender e criticar essa operação~\cite{sommerville}. São úteis na obtenção de detalhes mais aprofundados à requisitos que ainda estão gerais.

Os cenários serão utilizados para reduzir o nível de abstração das técnicas utilizadas anteriormente. Através dos mesmos, é esperado que os requisitos alcancem uma forma mais técnica e tangível, de forma a estar preparada para se tornar, enfim, um caso de uso.
