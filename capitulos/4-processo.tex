\chapter{Processo Escolhido}
\label{chosen-process}

\section{Imagem do Processo a ser Utilizado}

\section{Descrição das Atividades do Processo}

\subsection{Analisar o Problema}

\subsubsection{Identificar Envolvidos}

\textbf{Descrição}: Identificar quem são todas as pessoas envolvidas no processo de desenvolvimento\\
\textbf{Tarefas}: Entrevistas\\
\textbf{Resultados esperados do MPS-BR}: Nenhum\\
\textbf{Atividade da Engenharia de Requisitos}: Elicitação de Requisitos e Documentação\\
\textbf{Artefatos de entrada}: Nenhum\\
\textbf{Artefatos gerados ou alterados}: Documento especificando todos os \textit{stakeholders} e atores do sistema

\subsubsection{Entender o Contexto da Empresa}

\textbf{Descrição}: Compreender como é o funcionamento da empresa e seu atual processo\\
\textbf{Tarefas}: Entrevistas, Observações e Cenários\\
\textbf{Resultados esperados do MPS-BR}: Nenhum\\
\textbf{Atividade da Engenharia de Requisitos}: Elicitação de requisitos e Documentação\\
\textbf{Artefatos de entrada}: Nenhum\\
\textbf{Artefatos gerados ou alterados}: Documento explicando resumidamente o processo da empresa

\subsubsection{Entender Regras de Negócios}

\textbf{Descrição}: Especificar quais são as regras que regem o modo como a empresa opera\\
\textbf{Tarefas}: Entrevistas e cenários\\
\textbf{Resultados esperados do MPS-BR}: Nenhum\\
\textbf{Atividade da Engenharia de Requisitos}: Elicitação de Requisitos e Documentação\\
\textbf{Artefatos de entrada}: Nenhum\\
\textbf{Artefatos gerados ou alterados}: Documento de especificação de regras de negócios

\subsubsection{Desenvolver Documento de Visão}

\textbf{Descrição}: Condensar as informações levantadas nos passos anteriores e novas informações em uma visão geral sobre o produto a ser desenvolvido\\
\textbf{Tarefas}: Entrevistas, observação e cenários\\
\textbf{Resultados esperados do MPS-BR}: Nenhum\\
\textbf{Atividade da Engenharia de Requisitos}: Elicitação de Requisitos e Documentação\\
\textbf{Artefatos de entrada}: Documento de contexto, regras de negócio e de \textit{stakeholders}\\
\textbf{Artefatos gerados ou alterados}: Documento de Visão

\subsection{Compreender Necessidade dos Envolvidos}

\subsubsection{Identificar Características do Produto}

\textbf{Descrição}: Identificar as características do produto com alto nível de abstração\\
\textbf{Tarefas}: Entrevistas, observações e cenários\\
\textbf{Resultados esperados do MPS-BR}: GRE-1 e DRE-1\\
\textbf{Atividade da Engenharia de Requisitos}: Análise e Negociação\\
\textbf{Artefatos de entrada}: Temas de Investimento e Solicitação dos Envolvidos\\
\textbf{Artefatos gerados ou alterados}: Relato sobre o Produto, Requisitos Funcionais e Requisitos Não Funcionais

\subsubsection{Levantar RF e RNF}

\textbf{Descrição}: Identificar as características funcionais e não funcionais do produto\\
\textbf{Tarefas}: Entrevistas e refinar atributos dos requisitos\\
\textbf{Resultados esperados do MPS-BR}: GRE-1, DRE-1, DRE-7\\
\textbf{Atividade da Engenharia de Requisitos}:  Elicitação, Análise e Negociação, Documentação e Gerência de Requisitos\\
\textbf{Artefatos de entrada}: Temas de Investimento e Solicitação dos Envolvidos\\
\textbf{Artefatos gerados ou alterados}: Requisitos Funcionais, Requisitos Não Funcionais e Plano de Gerência de Requisitos

\subsubsection{Identificar Casos de Uso}

\textbf{Descrição}: Capturar os requisitos do sistema que agregarão valor ao cliente\\
\textbf{Tarefas}: Identificar atores e Identificar Requisitos Funcionais\\
\textbf{Resultados esperados do MPS-BR}: GRE-3 e DRE-4\\
\textbf{Atividade da Engenharia de Requisitos}: Verificação e Validação\\
\textbf{Artefatos de entrada}: Requisitos Funcionais\\
\textbf{Artefatos gerados ou alterados}: Modelo de Casos de Uso e Diagrama de Casos de Uso

\subsubsection{Elaborar Especificação Suplementar}

\textbf{Descrição}: Capturar os requisitos do sistema que não são identificados imediatamente no modelo de casos de uso\\
\textbf{Tarefas}: Identificar Requisitos Não Funcionais\\
\textbf{Resultados esperados do MPS-BR}: GRE-3, DRE-3, DRE-4 e DRE-7\\
\textbf{Atividade da Engenharia de Requisitos}: Verificação e Validação\\
\textbf{Artefatos de entrada}: Requisitos Não Funcionais\\
\textbf{Artefatos gerados ou alterados}: Especificação Suplementar

\subsubsection{Modelar Casos de Uso}

\textbf{Descrição}: Com base nos requisitos do \textit{software}, são criados os Casos de Uso e respectivos Diagramas de Caso de Uso e seus Casos de Teste\\
\textbf{Tarefas}: Descrever os Casos de Uso bem como seus diagramas e Casos de Teste\\
\textbf{Resultados esperados do MPS-BR}: GRE-2, DRE-2, DRE-3, DRE-5 e DRE-6\\
\textbf{Atividade da Engenharia de Requisitos}: Documentação\\
\textbf{Artefatos de entrada}: Requisitos Funcionais\\
\textbf{Artefatos gerados ou alterados}: Especificação de Requisitos de \textit{Software} e Casos de Teste

\subsubsection{Validar com o Cliente}

\textbf{Descrição}: Valida com o cliente se os requisitos elicitados e os casos de uso documentados condizem com as suas expectativas\\
\textbf{Tarefas}: Validar com o cliente os Requisitos Funcionais, Não Funcionais e Casos de Uso\\
\textbf{Resultados esperados do MPS-BR}: GRE-2, GRE-4, GRE-5 e DRE-8\\
\textbf{Atividade da Engenharia de Requisitos}: Verificação e Validação\\
\textbf{Artefatos de entrada}: Requisitos Funcionais, Não Funcionais e Casos de Uso\\
\textbf{Artefatos gerados ou alterados}: Reinicia o ciclo de desenvolvimento caso seja reprovado pelo cliente. Em cenário de aprovação, nada é alterado

\subsection{Planejar Iteração}

\subsubsection{Refinar Cronograma}

\textbf{Descrição}: Rever e atualizar o cronograma do projeto\\
\textbf{Tarefas}: Manter o cronograma consistente com o andamento do projeto\\
\textbf{Resultados esperados do MPS-BR}: GRE-4\\
\textbf{Atividade da Engenharia de Requisitos}: Gerência de Requisitos\\
\textbf{Artefatos de entrada}: Cronograma de Projeto\\
\textbf{Artefatos gerados ou alterados}: Cronograma de Projeto atualizado

\subsubsection{Priorizar Casos de Uso}

\textbf{Descrição}: Escolher quais Casos de Uso serão implementados\\
\textbf{Tarefas}: Escolher os Casos de Uso a serem implementados de acordo com as prioridades e pré-requisitos\\
\textbf{Resultados esperados do MPS-BR}: DRE-2\\
\textbf{Atividade da Engenharia de Requisitos}: Gerência de Requisitos\\
\textbf{Artefatos de entrada}: Casos de Uso e Casos de Teste\\
\textbf{Artefatos gerados ou alterados}: Casos de Uso priorizados serão descritos no Plano de Iteração

\subsubsection{Produzir Plano de Iteração}

\textbf{Descrição}: Descrever quais Casos de Uso foram priorizados para a iteração\\
\textbf{Tarefas}: Produzir o artefato de Plano de Iteração\\
\textbf{Resultados esperados do MPS-BR}: DRE-2\\
\textbf{Atividade da Engenharia de Requisitos}: Gerência de Requisitos\\
\textbf{Artefatos de entrada}: Casos de Uso e Casos de Teste\\
\textbf{Artefatos gerados ou alterados}: Plano de Iteração

\subsection{Implementação}

\subsubsection{Refinar Casos de Uso Priorizados}

\textbf{Descrição}: Validar os casos de uso com o cliente para garantir que todos os atributos dos casos de uso estão de acordo com a necessidade do cliente\\
\textbf{Tarefas}: Validar Casos de Uso com o cliente e detalhar Casos de Uso\\
\textbf{Resultados esperados do MPS-BR}: GRE-4, DRE-3, DRE-4, DRE-7, DRE-8\\
\textbf{Atividade da Engenharia de Requisitos}: Análise e Negociação\\
\textbf{Artefatos de entrada}: Diagrama de Casos de Uso e Modelo de Casos de Uso\\
\textbf{Artefatos gerados ou alterados}: Diagrama de Casos de Uso, Modelo de Casos de Uso, Especificação de Requisitos de \textit{Software}

\subsubsection{Implementar Casos de Uso}

\textbf{Descrição}: Implementar os Casos de Uso que foram descritos na Especificação de Requisitos de \textit{Software}\\
\textbf{Tarefas}: Implementar parte da solução, com base na Especificação de Requisitos de \textit{Software} e validar implementação de Caso de Uso com o cliente\\
\textbf{Resultados esperados do MPS-BR}: Nenhum\\
\textbf{Atividade da Engenharia de Requisitos}: Nenhuma\\
\textbf{Artefatos de entrada}: Diagrama de Casos de Uso, Modelo de Caso de Uso e Especificação de Caso de Uso\\
\textbf{Artefatos gerados ou alterados}: Nenhum

\subsubsection{Gerenciar Desenvolvimento}

\textbf{Descrição}: Verificar se as atividades da ER estão de acordo com a visão do cliente\\
\textbf{Tarefas}: Revisar os requisitos e gerenciar dependências\\
\textbf{Resultados esperados do MPS-BR}: GRE-3, GRE-4, GRE-5\\
\textbf{Atividade da Engenharia de Requisitos}: Gerenciar Requisitos\\
\textbf{Artefatos de entrada}: Diagrama de Casos de Uso, Modelo de Casos de Uso e Documento de Visão\\
\textbf{Artefatos gerados ou alterados}: Nenhum

\subsubsection{Entregar Produto}

\textbf{Descrição}: Entregar parte da solução para o cliente, indicando o fim do ciclo de vida do projeto\\
\textbf{Tarefas}: Revisar os requisitos e gerenciar dependências\\
\textbf{Resultados esperados do MPS-BR}: Nenhum\\
\textbf{Atividade da Engenharia de Requisitos}: Nenhuma\\
\textbf{Artefatos de entrada}: Diagrama de Casos de Uso, Modelo de Casos de Uso e Documento de Visão\\
\textbf{Artefatos gerados ou alterados}: Nenhum

\section{Papéis}

\subsection{Gerente de Controle de Mudanças}

\textbf{Definição}
\begin{itemize}
\item Supervisiona o controle de mudanças, geralmente formado por um conjunto entre todos os envolvidos
\end{itemize}

\textbf{Responsável por}
\begin{itemize}
\item Estabelecer Processo de Controle de Mudanças
\item Revisar Solicitação de Mudanças
\item Confirmar se há duplicidade entre as solicitações
\end{itemize}

\textbf{Artefato gerado ou alterado}
\begin{itemize}
\item Solicitação de Mudança
\end{itemize}

\subsection{Gerente de Projeto}

\textbf{Definição}
\begin{itemize}
\item Aloca recursos, ajusta prioridades, coordena interações com clientes e mantém a equipe focada em uma meta em específico
\end{itemize}

\textbf{Responsável por}
\begin{itemize}
\item Caso de Negócio
\item Métricas do Projeto
\item Iniciar o Projeto
\end{itemize}

\textbf{Artefato gerado ou alterado}
\begin{itemize}
\item Caso de Negócio
\end{itemize}

\subsection{Analista de Processos de Negócios}

\textbf{Definição}
\begin{itemize}
\item Lidera e coordena a modelagem de casos de uso de negócios, definindo e delimitando a organização que está sendo modelada
\end{itemize}

\textbf{Responsável por}
\begin{itemize}
\item Estabelece quais são os atores de negócio e Caso de Uso de negócio
\item Capturar vocabulário comum
\item Estruturar Modelo de Casos de Uso de negócio
\end{itemize}

\textbf{Artefato gerado ou alterado}
\begin{itemize}
\item Glossário
\item Especificação Suplementar de negócios
\item Modelo de Caso de Uso de negócios
\end{itemize}

\subsection{Analista de Sistemas}

\textbf{Definição}
\begin{itemize}
\item Lidera e coordena a identificação de requisitos e a Modelagem de Casos de Uso, delimitando o sistema e definindo suas funcionalidades
\end{itemize}

\textbf{Responsável por}
\begin{itemize}
\item Desenvolver Plano de Gerência de Requisitos
\item Desenvolver Visão
\item Identificar Solicitação dos Envolvidos
\item Identificar Atores e Casos de Uso
\end{itemize}

\textbf{Artefato gerado ou alterado}
\begin{itemize}
\item Plano de Gerência de Requisitos
\item Visão
\item Atributos de Requisitos
\item Solicitação dos Envolvidos
\end{itemize}

\subsection{\textit{Designer} de Negócios}

\textbf{Definição}
\begin{itemize}
\item Detalha a especificação de uma parte da organização, descrevendo o fluxo de trabalho de um ou de vários Casos de Uso de negócios
\end{itemize}

\textbf{Responsável por}
\begin{itemize}
\item Detalhar Casos de Uso de negócios
\end{itemize}

\textbf{Artefato gerado ou alterado}
\begin{itemize}
\item Casos de Uso de negócios
\end{itemize}

\subsection{Especificador de Requisitos}

\textbf{Definição}
\begin{itemize}
\item Detalha a especificação de uma parte da funcionalidade do sistema, descrevendo os requisitos de um ou vários Casos de Uso 
\end{itemize}

\textbf{Responsável por}
\begin{itemize}
\item Detalhar Casos de Uso
\item Detalhar Requisitos de \textit{Software}
\end{itemize}

\textbf{Artefato gerado ou alterado}
\begin{itemize}
\item Casos de Uso
\item Especificação de Requisitos
\end{itemize}
