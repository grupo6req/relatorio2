\chapter{Refatoração}
\label{refactor}

Após a primeira apresentação, houve um \textit{feedback} construtivo por parte da professora acerca dos erros encontrados no projeto, como a forma de rastreabilidade explicitada. Além disso, com o decorrer das aulas, houve um amadurecimento dos membros da equipe, que permitiram analisar alguns poucos pontos que não eram benéficos para o desenvolvimento do trabalho. Neste capítulo, pretende-se detalhar esses pontos e as soluções encontradas para cada um.

\section{Processo}

É importante comentar que o processo utiliza atividades que geram documentos que, a maioria das vezes, servirão de insumo para documentos posteriores. É assim na \textit{line} 1 do documento, em que o artefato gerado na atividade de "Identificar Envolvidos", que seria um documento especificando todos os \textit{stakeholders}, é utilizado no tópico 3 do documento de visão, "Descrição dos Envolvidos e dos Usuários". Isso foi feito para que houvesse uma maior modularização no desenvolvimento do projeto.

Na \textit{line} 2 do processo, foram encontrados algumas informações incorretas que resultaram nas seguintes correções:

\begin{itemize}
\item Nas atividades de "Levantar RF" e "Levantar RNF", foram gerados a Especificação de Requisitos de \textit{Software} e os requisitos não funcionais, respectivamente. O último serve de insumo para a atividade "Elaborar Especificação Suplementar". Nessa atividade, também foi gerado o Plano de Gerência de Requisitos.
\item A atividade de "Identificar Casos de Uso" foi identificada erroneamente na imagem do processo como "Especificar Casos de Uso". Além disso, a saída dessa atividade não inclui o Modelo de Casos de Uso, como especificado, apenas o Diagrama de Casos de Uso, que foi especificado corretamente.
\item Na atividade de Modelar Casos de Uso, foi gerado a Especificação de Caso de Uso, que complementada com o Diagrama de Casos de Uso elaborado anteriormente, forma o Modelo de Caso de Uso. A informação presente no primeiro relatório de que seria gerado a Especificação de Requisitos de \textit{Software} nessa atividade foi errônea, visto que esse documento foi gerado em uma atividade anterior supracitada. Também é gerado nessa atividade o artefato de Casos de Teste, como foi previsto.
\end{itemize}

Por fim, a imagem do processo foi atualizada de acordo com as mudanças descritas acima.
